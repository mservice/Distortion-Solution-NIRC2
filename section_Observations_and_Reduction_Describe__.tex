
\section{New Distortion Model}
The goal of this work is to characterize the static component of the NIRC2 narrow field camera.  This updates the solution of \cite{Yelda_2010}, and uses a similar approach.  There are two primary differences between our methodology and theirs; we choose to use bivariate Legendre polynomials as the fitting basis for the model and we use an iterative approach to create new distortion free reference frames from the NIRC2 measurements.  This allows us to mitigate the effect of stellar motion between the 2006 HST measurements and the 2015 NIRC2 measurements.
\subsection{Constructing the Model}
To model the geometric optical distortion of the NIRC2 one must first account for the fact that HST data does not suffer from DAR, while ground based images do.  DAR causes images to be compressed along the zenith, which in turn causes measured stellar position separations to appear smaller than they truly are.  Ideally, we would simply magnify the NIRC2 positions along the zenith angle to correct this issue, however,  this is impossible when the distortion solution is unknown.  As we do not know the distortion solution, we do not know the true positions of the stars, and thus cannot directly correct for the DAR.  Instead we apply a compression to the HST positions on a frame by frame basis to create positions that can be matched to the distorted NIRC2 positions measurements. This method precisely mirrors the work of \cite{Yelda_2010}, and uses the prescription to account for DAR given by \cite{Gubler}
We create final position catalogs for each pointing by combining the position measurements from all images taken at that pointing  Stars that appear in at least 2 images are kept in the stacked catalogs.  The rms errors from these stacks are the positional uncertainties (Figure \ref{cent_precision}), and they are used to weight the fits.

After this correction, sources are matched between the HST and NIRC2 catalogs.  The deltas between theses positions are the data that will be fit   Figures \ref{quiver_apr} and \ref{quiver_may} show the measured positional differences.  As expected, the deltas show clear spatial correlations, however there are also large outliers.  These points are eliminated prior to fitting the data by clipping all 3$\sigma$ outliers in both X and Y in 205x205 pixel bins.  After clipping, this leaves a total of 4779 delta measurements in April from 609 unique stars and 3336 delta measurements in May from 394 unique stars. 

The data are fit with multivariate Legendre Polynomials (cite Stestson).  We select a $6^{th}$ order polynomials as it shows smaller residuals than the $5^{th}$ order, but increasing to $7^{th}$ does not.  Figure \ref{dist_x} and \ref{dist_y} shows fits to distortion solution from April.  The model is fit by allowing the coefficients from equations \ref{eq:legx} and \ref{eq:legy} to vary, where $L_{n}$ are the $n^{th}$ Legendre polynomials.  
\begin{equation}
x^{'} = a_{0} + a_{1}L_{1}(x)L_{0}(y) +  a_{2}L_{0}(x)L_{1}(y) + a_{3}L_{2}(x)L_{0}(y)+...
\label{eq:legx}
\end{equation}
\begin{equation}
y^{'} = b_{0} + b_{1}L_{1}(x)L_{0}(y) +  b_{2}L_{0}(x)L_{1}(y) + b_{3}L_{2}(x)L_{0}(y)+...
\label{eq:legy}
\end{equation}
This first round of fits are weighted by the positional uncertainties in both the HST and NIRC2 data.  This method of fitting the distortion has a major flaw, namely that the HSt images were observed in 2006, while the NIRC2 images were taken in 2015.   This can be solved by creating a reference catalog by applying the above distortion solution (fit using the differences between HST and NIRC2 positions) and then stacking the NIRC2 catalogs to create a new distortion free reference catalog.  We then use the new catalog instead of the HST measurements and repeat the entire fitting procedure.  Expeirmentation demonstrated that 10 iterations were sufficient to reach a stable solution, so we use those solutions as our final distortion model.  



