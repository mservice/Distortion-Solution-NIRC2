\section{Observations and Reduction}
-Describe Observations (2 epochs around AO adjustment)
-Data quality Strehl
-Starfinder
-Cite previous GC papers, keep short as reasonable

\section{New Distortion Model}
The goal of this work is to characterize the static component of the NIRC2 narrow field camera.  This updates the solution of \cite{Yelda_2010}, and uses the same approach.  The difference between positions in a nominally distortion free frame and the observed stellar positions in NIRC2 are used to fit a single surface.  This surface forms our model of the time-averaged distortion  in the telescope and detector system.

\subsection{Constructing the Model}
To model the geometric optical distortion of the NIRC2 one must first account for the fact that HST data does not suffer from DAR, while ground based images do.  DAR causes images to be compressed along the zenith, which in turn causes measured stellar position separations to appear smaller than they truly are.  Ideally, we would simply magnify the NIRC2 positions along the known zenith angle to correct this issue, however,  this is impossible when the distortion solution is unknown.  Distoriton corrected positions must be calculated prior to applying a DAR correction, so instead we simply apply a compression to the HST positions on a frame by frame basis. This method precisely mirrors the work of \cite{Yelda_2010}, and uses the prescription to account for DAR given by \cite{Gubler}
We create final position catalogs for each pointing by combining the position measurements from all images taken at that pointing  Stars that appear in at least 2 images are kept in the stacked catalogs.  The rms errors from these stacks are the positional uncertainties (Figure \ref{cent_precision}), and they are used to weight the fits.

Each NIRC2 star list is matched with the ACS stellar positions, and the NIRC2 positions are used as a reference into which the ACS positions are transformed.  A four parameter transformation is used; this accounts for translation, rotation and global plate scale.  Before this transformation can be performed the ACS position measurements have are DAR corrected to match positions as measured at the telescope after traversing the atmosphere.  We add the effect to the space data instead of correcting the ground data because the distortion solution is unknown, so we do not know the true positions of the stars.  After this correction, sources are matched between the two catalogs and used to create the distortion model.  This gives a total of 4873(FIXME) matched position measurements in April from 619(FIXME) unique stars  and 3223(FIXME) position measurements in May from 403(FIXME) unique stars.  Figures \ref{quiver_apr} and \ref{quiver_may} show the measured positional differences.  As expected, the deltas show clear spatial correlations, however there are also large outliers.  These points are eliminated prior to fitting the data by clipping all 3$\sigma$ outliers in both X and Y in 205x205 pixel bins.  


The data are fit with multivariate Legendre Polynomials (cite Stestson).  We select a $6^{th}$ order polynomials as it shows smaller residuals than the $5^{th}$ order, but increasing to $7^{th}$ does not.  Figure \ref{dist_x} and \ref{dist_y} shows fits to distortion solution from April.  The model is fit by allowing the coefficients from equations \ref{eq:legx} and \ref{eq:legy} to vary, where $L_{n}$ are the $n^{th}$ Legendre polynomials.  
\begin{equation}
x^{'} = a_{0} + a_{1}L_{1}(x)L_{0}(y) +  a_{2}L_{0}(x)L_{1}(y) + a_{3}L_{2}(x)L_{0}(y)+...
\label{eq:legx}
\end{equation}
\begin{equation}
y^{'} = b_{0} + b_{1}L_{1}(x)L_{0}(y) +  b_{2}L_{0}(x)L_{1}(y) + b_{3}L_{2}(x)L_{0}(y)+...
\label{eq:legy}
\end{equation}
This first round of fits are weighted by the positional uncertainties in both the HST and NIRC2 data.  This first solution is penalized by the true proper motions of the stars resulting in positional changes between 2006(DOUBLE CHECK) and 2015.  To mitigate this, we create a reference catalog from all the NIRC2 images and then repeat the entire fitting procedure.  This is done until the solution is stable between consecutive iterations.  


