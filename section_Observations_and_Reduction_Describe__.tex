\section{Observations and Reduction}
-Describe Observations (2 epochs around AO adjustment)
-Data quality Strehl
-Starfinder
-Cite previous GC papers, keep short as reasonable

\section{New Distortion Model}
The goal of this work is to characterize the static component of the NIRC2 narrow field camera.  This updates the solution of \cite{Yelda_2010}, and uses the same approach.  The difference between positions in a nominally distortion free frame and the observed stellar positions in NIRC2 are used to fit a single surface.  This surface forms our model of the time-averaged distortion in the telescope and detector system.

\subsection{Constructing the Model}
Each NIRC2 star list is first matched to the 3 or 4 other exposures taken at that position.  Stars that appear in at least 2 images are kept in the stacked catalogs.  The rms errors from these stacks are the positional uncertainties for the NIRC2 dataset (Figure \ref{cent_precision}).  These uncertainties peak at a value of 

Each NIRC2 star list is matched with the ACS stellar positions, and the NIRC2 positions are used as a reference into which the ACS positions are transformed.  A four parameter transformation is used; this accounts for translation, rotation and global plate scale.  Only sources that are matched between the two lists are used to create the distortion model.    This gives a total of ~5000 matched position measurements in April and ~4000 position measurements in May (note numbers got divided by ~4 because of the stacking).  



The deltas between the ACS and NIRC2 positions are spatially correlated, however there are large outliers.  These points are eliminated prior to fitting the data by clipping all 3$\sigma$ outliers in both X and Y in 205x205 pixel bins.  

Bivatiate B-splines and Legendre polynomials are fit to the delta map.  We choose the two fitting methods as a check that the final solution is not sensitive to the choice.  Figure \ref{dist_x} and \ref{dist_y} shows the two distortion models, once created using sixth order Legnedre polynomials and the other generated using a spline with smoothing factor of 10000.  The Legendre fits fits of the coefficients from equations \ref{eq:legx} and \ref{eq:legy} , where $L_{n}$ are the $n^{th}$ Legendre polynomials.  
\begin{equation}
x^{'} = a_{0} + a_{1}L_{1}(x)L_{0}(y) +  a_{2}L_{0}(x)L_{1}(y) + a_{3}L_{2}(x)L_{0}(y)+...
\label{eq:legx}
\end{equation}
\begin{equation}
y^{'} = b_{0} + b_{1}L_{1}(x)L_{0}(y) +  b_{2}L_{0}(x)L_{1}(y) + b_{3}L_{2}(x)L_{0}(y)+...
\label{eq:legy}
\end{equation}
Both of these fits are weighted by the positional uncertainties in both the HST and NIRC2 data(note,this isn't yet true, but will be soon). Figure \ref{spline_resid} show the residual between the spline fit and the data.  The .6 pixels of residual to the fit is much larger than the positional uncertainties.  This may be caused by the functional form of the fit not having the degrees of freedom to fit the data, or the uncertainties could be severely underestimated.   These residual between these models has a standard deviation of $\sim$ .1 pixels.
