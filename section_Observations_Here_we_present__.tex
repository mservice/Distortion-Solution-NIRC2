\section{Observations}
Here we present the observations and reduction carried out for this work.  This reduction procedure mirrors the work of \cite{Yelda_2010} which derived a similar distortion solution.  
\subsection{M53}
Characterizing the geometric optical distortion in the NIIRC2 camera requires two sets of positions measurements.  Ideally positions measurements at the camera focal plane could be compared to a distortion free frame.  As no such reference frame exists, we use observations of M53 made with the Advanced Camera for Surveys Wide Filed Channel (ACS) on the Hubble Space Telescope(HST).  the static distortion in the camera has been corrected to the $\sim$0.5 mas level.

M53 was observed on April 2, 2015 and May 5, 2015 using the laser AO system in the W.M. Keck II 10 m telescope with the facility near infrared camera NIRC2 (PI: K.Matthews).  All of these observations were taken using the narrow field camera (10" x 10") through the K' filter ($\lambda_{0} = $ 2.12 $\mu$m, $\Delta\lambda =$ 0.35 $\mu$m).  THe AO bench was realigned  in late April to improve the overall performance.  A unintended side effect of this procedure was that the geometric distortion of the system changed at that point.  Therfore, we treat these two data sets independently to derive separate distortion solutions for the pre- and post- realignment data 

M53 was observed at two P.A.s during each night of observations with 40 positional offsets in April and 28 positional offsets in May.  Three identical exposures were taken at each unique pointing.  This observing strategy allows a given star to be measured at many locations on the chip.

The NIRC2 images are reduced and calibrated in the standard way (see \cite{Yelda_2010} for details).  After reducing the images, Starfinder \citep{Diloatti} is used on each exposure with a correlation value of 0.8 to generate a catalog of stellar measurements.  Position measurements from identical images are then averaged to create final stellar position measurements for each pointing.  The uncertainty is estimated from the rms of the three measurements per stars.  Note that there is no brightness or positional error cut applied to the data.





